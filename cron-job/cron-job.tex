\documentclass[12pt]{article}
\usepackage{verbatim}
\usepackage[dvips]{epsfig}
\usepackage{color}
\usepackage{url}
\usepackage[colorlinks=true]{hyperref}

\begin{document}

\section*{Neurospaces: Documentation}

{\bf Related Documentation:}
% start: heterarch-tagreplaceitems related-do-nothing
% end: heterarch-tagreplaceitems related-do-nothing

\section*{Heterarch cron jobs}

The Neurospaces / TWS Heterarch supports the automated construction of
websites using cron jobs (see also
\href{http://www.genesis-sim.org/userdocs/neurospaces-cron/neurospaces-cron.html}{the
  Genesis-3 cron job manual} and
\href{http://www.genesis-sim.org/userdocs/userdocs-cron/userdocs-cron.html}{the
  Genesis-3 Documentation System cron job manual}).  Cron jobs can be
associated with specific packages that have previously been
instantiated with the command {\tt neurospaces\_new\_component
  --heterarch}.


\subsection*{To create a new cron job}

\begin{enumerate}
\item Create a new user specific for the cron job with the {\tt
    adduser} unix shell command.

  Note that the repository of new packages can be served from their
  own account.  If that is the case, make sure that the cron job does
  not use this same account as that may lead to monotone database lock
  problems.

\item If the cron job is going to reinstall any of the packages, give
  the new user passwordless {\tt sudo} access: {\tt echo 'joeuser
    ALL=(ALL) NOPASSWD:ALL' >> /etc/sudoers}

  {\it For most configurations only the heterarch should be given
    permission to upgrade the developer package and the heterarch.
    However, the new package should have permissions for its own
    installation, making available the commands for synchronization
    and build of the website.  These commands can be installed
    manually when following the procedure outlined here.}

\item Put the new user into the {\tt heterarch} group to make sure it
  has write access to the html output directory.
\item Create a suitable configuration file for the cron job.  Sample
  contents:
\begin{verbatim}
---
MAILTO: hugo.cornelis@gmail.com, allan.coop@gmail.com
MAILFROM: hfof_cron
HTMLDIR: /var/www/html
OUTPUTDIR: /tmp
URL: http://91.183.94.6/html/hfof
LOGROTATE: 100
\end{verbatim}
\item Change your role to the new user.
\item For each package that should be published run the following
  commands:
  \begin{enumerate}
  \item Run the command '{\tt neurospaces\_create\_directories
      <package-name>}' for each package that should be published.
  \item {\tt neurospaces\_pull <package-name>} {\it note: this step should
      become optional in the future}
  \item {\tt neurospaces\_upgrade <package-name>}

    This gives you a {\tt <package-name>\_cron} command for each
    package installed.  You can test it in isolation by running it
    from the command line.  For reasons of performance and to avoid
    race conditions, it is recommended to configure one such command
    to build all the packages.
  \end{enumerate}
\item Add {\tt <package-name>\_cron} to the {\tt /etc/crontab} file, for
  instance for execution every hour:

\begin{verbatim}
1 *     * * *   <user-name>  ( cd /home/<user-name>/neurospaces_project/<package-name>/source/snapshots/0 && /usr/local/bin/<package-name>_cron --config /home/<username>/<package-name>_cron.yml )
\end{verbatim}
\end{enumerate}


%\subsubsection*{Figures}

%Figures can easily be added with the following code:

%\begin{verbatim}
%\begin{figure}[h]
%   \centering
%   \includegraphics[scale=0.6]{figures/dummyfig.eps}
%   \caption{{\bf A Dummy Figure:} Example of Latex code to incorporate
%      a figure into documentation.}
%   \label{fig:df-1}
%\end{figure}
%\end{verbatim}

%\begin{figure}[h]
%  \centering
%   \includegraphics[scale=0.5]{figures/dummyfig.eps}
%\caption{{\bf A Dummy Figure:} Example of \LaTeX\,\,\,code to incorporate a figure into documentation.}
%  \label{fig:df-2}
%\end{figure}

\bibliographystyle{plain}
\bibliography{../tex/bib/g3-refs.bib}

\end{document}
