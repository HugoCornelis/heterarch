\documentclass[12pt]{article}
\usepackage{verbatim}
\usepackage[dvips]{epsfig}
\usepackage{color}
\usepackage{url}
\usepackage[colorlinks=true]{hyperref}
\usepackage{enumitem} % [noitemsep,nolistsep] compact itemize/enumerate etc
\usepackage[margin=2cm]{geometry}

% Typeface
\usepackage[condensed,math]{iwona}
\usepackage[T1]{fontenc}

\begin{document}

\section*{HETERARCH: Documentation}

{\bf Related Documentation:}
% start: userdocs-tag-replace-items related-do-nothing
% end: userdocs-tag-replace-items related-do-nothing

\section*{Heterarch}

\subsection*{Preamble}

The Heterarch Content Management System ({\bf \small HCMS}) is an open source software platform that supports distributed personalized digital content broadcasting. It is developed and maintained by {\bf Neurospaces\,Development} in collaboration with {\bf Three\,Way\,Street}.

%E-publication of the physical book {\bf{\emph{Head\,Full\,of\,Flames}}} ({\bf \small HFoF}) is used as a proof of concept for our novel approach to the structure and management of online content for the purpose of sophisticated online e-publication and community building.
One distinguishing feature of {\bf \small HCMS} is the use of modern techniques for the organization, structure, and recovery of digital content and/or identity. Our approach is heterarchical and provides a powerful alternative to more traditional hierarchical methods of software implementation and content management. It is the heterarchical nature of the {\bf \small HCMS} that allows content recovery to be defined within an educational learning paradigm that ranges from ``search'', through ``explore'' to ``discover'', ``understand'', and ``know''.

This fractionation of cognitive activity has many benefits for both developers and users. In particular, for the developer, it allows a far greater degree of control over data flows and thus automation of the many non-trivial but time consuming processes associated with sophisticated distributed personalized digital content broadcasting and management. For the user, it reconfigures the traditional hierarchical online experience of the internet into a powerful heterarchcal information-based learning environment.

\subsection*{Document Purpose}

This document provides an introductory overview of {\bf \small HCMS} functionality.

\subsection*{Definitions}

There are two domains of human activity associated with the {\bf \small HCMS}. Each of these domains is divided into two sub-domains:

\begin{enumerate}[noitemsep,nolistsep] 
\item {\bf Developer}\\
The two sub-domains of developer functionality include:
	\begin{description}[noitemsep,nolistsep] 
	\item[A.] {\bf Content Developer}\\
	A content developer must obtain authorization from a repository owner to be able to directly access a given 
	repository.
		\begin{description}[noitemsep,nolistsep] 
		\item {\bf Authorized Content Developer}
			\begin{itemize}[noitemsep,nolistsep] 
			\item By default, the owner(s) of a repository is/are (an) authorized person(s) and may be (an) authorized 
			content developer(s).
			\item The owner(s) of a repository can authorize any person the permissions required to access {\bf \small
			HCMS} functionality of a given repository for the purpose of creating, adding  and/or modifying content 
			owned by an authorized person.
			\item An authorized content developer is any person given access to a repository for the purpose of
			creating,  adding, and/or modifying content. %Two types of content developer are recognized:
		\end{itemize}
		\end{description}			
	\item[B.] {\bf Software Developer}\\
	A software developer extends {\bf \small HCMS} functionality by the creation and integration of new software and 
	functionality.
	\end{description}
\item {\bf User}\\
The two sub-domains of user include:
	\begin{description}[noitemsep,nolistsep] 
	\item[A.] {\bf Content User}\\
	A content user is any person who is not an Authorized Content Developer who accesses {\bf \small HCMS} content 
	through a internet browser or other web interface.
	\item[B.] {\bf Software User}\\
	A software user is any person who employs {\bf \small HCMS} software functionality to manage a given {\bf \small 
	HCMS} repository.
	\end{description}
\end{enumerate}

\subsection*{Installation}

Currently, there are two types of {\bf \small Heterarch} installation:

\begin{enumerate}[noitemsep,nolistsep] 
\item[(A)] {\bf User:} If you are a Content User, no software is required as you access the content of a {\bf\small Heterarch} repository with a browser of your choice.
\item[(B)] {\bf Developer}: If you are a Content or Software Developer, or a Software User, you must install {\bf\small Heterarch} on your local machine.\\
{\bf Note:} If you are a Developer you are usually also a Software User. For the purposes of this documentation Software Users are considered to be Developers.
\end{enumerate}
Typically, once your machine is correctly configured and any missing \href{../heterarch-dependencies/heterarch-dependencies.tex}{\bf Heterarch Software Dependencies} are installed, the installation process is highly automated.

\subsection*{Developer Installation}

The following links introduce and define the required installation steps:
\begin{enumerate}[noitemsep,nolistsep]
\item {\bf Developer Content:} This is a more limited subset of Developer Software. General installation instructions are located in \href{../heterarch-developer-content/heterarch-developer-content.tex}{\bf Developer\,Content}.\\
Such an installation synchronizes with the \href{../developer-repository/developer-repository.tex}{\bf Developer\,Repositories} and is under \href{../version-control/version-control.tex}{\bf Version\,Control}.
\item {\bf Developer Software:} Software tools and functionalities are significantly extended beyond that required for Developer Content. General installation instructions are located in \href{../heterarch-developer-software.tex}{\bf Developer\,Software}.\\ Such an installation also synchronizes with the \href{../developer-repository/developer-repository.tex}{\bf Developer\,Repositories} and is under \href{../version-control/version-control.tex}{\bf Version\,Control}.\\
\end{enumerate}

\subsection*{HCMS Access}

Access to a repository, content creation, content addition, and content modification are controlled by (a) repository owner(s).

\subsection*{System Structure}

We now describe content items and the way they are structured, implemented, and managed, and introduce the system functionality and its control within a {\bf \small HCMS} repository.

\subsubsection*{The HCMS Repository}

Each {\bf \small HCMS} installation generates a stand alone repository containing the software required to support {\bf \small HCMS} functionality, and manage and update repository content.\\
Content items can be either generated in or added to this repository by the repository owner or an authorized content developer.\\
Each content item may range in size from a single link to the addressable storage space known to a given {\bf \small HCMS}.\\
External content can be hyperlinked to content within a {\bf \small HCMS} repository.

\subsubsection*{Repository Location}

A {\bf \small HCMS} repository includes the directory structure of a given installation and the directory where content items are maintained. Following installation of a {\bf \small HCMS}, the default location of the repository is:

\begin{verbatim}
    ~/neurospaces_project/<repository-name>/source/snapshots/0
\end{verbatim}
The source materials for individual content items are located in uniquely named folders/directories in this repository (at the end of the path in directory/folder ``0''). One of the directories/folders in the repository is named\,{\it NewDocument}. This default content item and the files it contains provide the templates for the creation of a new content item. The files of interest are called\,{\it descriptor.yml}, {\it NewDocument.tex}, and the sub-directory {\it figures}.

\subsubsection*{Content Items}

An individual content item may have any size or any length. It may contain any number of photographs/illustrations and links to any other content located in either the same repository, another {\bf \small HCMS} repository, or elsewhere on the internet.\\
{\bf Importantly, the addition of any content to a workspace or repository is an acknowledgement by the authorized content developer of their ownership rights.}\\
For more detailed information see \href{../developer-content/developer-content.tex}{\bf Developer\,Content}.

%\bibliographystyle{plain}
%\bibliography{../tex/bib/g3-refs.bib}

\end{document}
