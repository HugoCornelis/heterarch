\documentclass[12pt]{article}
\usepackage{verbatim}
\usepackage[dvips]{epsfig}
\usepackage{color}
\usepackage{url}
\usepackage[colorlinks=true]{hyperref}
\usepackage{enumitem} % [noitemsep,nolistsep] compact itemize/enumerate etc
\usepackage[margin=2cm]{geometry}

% Typeface
\usepackage[condensed,math]{iwona}
\usepackage[T1]{fontenc}

% Remove spacing following headings, ie. implement \usepackage[compact]{titlesec}
%\titlespacing{\section}{0pt}{*0}{*0}
%\titlespacing{\subsection}{0pt}{*0}{*0}
%\titlespacing{\subsubsection}{0pt}{*0}{*0}

% Remove section numbering
\setcounter{secnumdepth}{0}

% Inter-repository linking
\newcommand{\heterarchxref}[2]{\href{#1}{#2}}

 % Generate a nicely located tilde
\newcommand{\mytilde}{\raise.17ex\hbox{$\scriptstyle\mathtt{\sim}$}}

\begin{document}

\section{Heterarch: Documentation}

{\bf Related Documentation:}
% start: userdocs-tag-replace-items related-do-nothing
% end: userdocs-tag-replace-items related-do-nothing

\section{Heterarch}

\subsection{Preamble}

The Heterarch Content Management System ({\bf \small HCMS}) is an open source software platform that supports distributed content management. It is developed and maintained by the \href{http://neurospaces.sourceforge.net/}{\bf Neurospaces\,Project} in collaboration with {\bf Three\,Way\,Street}. 

The distinguishing feature of the {\bf \small HCMS} is its use of modern techniques for the organization, structure, and recovery of digital content. Our approach is heterarchical and provides a powerful alternative to more traditional hierarchical methods of content management. It is the heterarchical nature of the {\bf \small HCMS} that allows content recovery to be defined as a ``discover'' paradigm, as opposed to a still widely used but more conceptually limited alternative, the ``search'' paradigm.

A change in perspective from search to discover has many benefits for both developers and users. In particular, for the developer, it allows a far greater degree of automation for the many non-trivial but time consuming processes associated with sophisticated content management. For the user, it provides a powerful information-based experience.

\subsubsection{Primary Functions}

One of the primary functions of the {\bf \small HCMS} is to support individuals/social groups/communities through electronic publication and commercialization of the digital content they create either individually or collectively.

Along with image (static and moving) and sound content, we distinguish three types of publication:
\begin{description}[noitemsep,nolistsep]
\item {\bf P-Publication (p-pub):}\\
This is the traditional paper based publication.
\item {\bf E-Book (e-pub)}:\\
This is the contemporary experience of digital publication. There are a wide variety of formats. The {\bf \small HCMS} recognizes the reflowable content EPUB format for dedicated readers such as Kindle, Nook, iPad, Sony Reader, and Kobo; and the fixed PDF format.
\item {\bf HCMS (d-pub):}\\
The paradigm breaking distributed digital publication and discovery platform offered by {\bf \small Three Way Street}.
\end{description}

\subsection{Document Purpose}

This document provides an introductory overview of the {\bf \small HCMS} d-pub functionality.

\subsection{Definitions}

There are two classes of human functionality associated with the {\bf \small HCMS}. Each of these classes is divided into two further sub-classes:

\begin{enumerate}[noitemsep,nolistsep] 
\item {\href{../hfof-developer/hfof-developer.tex}{\bf Developers:}}\\
The two sub-classes of developer include:
	\begin{description}[noitemsep,nolistsep] 
	\item {\bf Content Developer}\\
	A content developer must obtain authorization from a repository owner before being able to directly access a given repository.
		\begin{description}[noitemsep,nolistsep] 
		\item {\bf Authorized Content Developer:}\\
		An authorized content developer is any person given access to a repository for the purpose of creating, adding, and/or modifying content. Two types of content developer are recognized:
			\begin{description}[noitemsep,nolistsep] 
			\item {\bf Repository owner(s):}\\
			By default, the owner(s) of a repository is/are authorized content developer(s).
			\item {\bf Authorized content developer(s):}\\
			The owner(s) of a repository can authorize any person who is not a repository owner the permissions required to access {\bf \small HCMS} functionality of the given repository for the purpose of creating and/or adding content owned by the authorized person.
			\end{description}
		\end{description}			
	\item {\bf Software Developer}\\
	A software developer extends {\bf \small HCMS} functionality by the creation and integration of new software.
	\end{description}
\item {\bf Users:}\\
The two sub-classes of user include:
	\begin{description}[noitemsep,nolistsep] 
	\item {\bf Content User}\\
	A content user is any person who is not an Authorized Content Developer who accesses {\bf \small HCMS} content through a web browser.
	\item {\bf Software User}\\
	A software user is any person who employs {\bf \small HCMS} software functionality to manage a given {\bf \small HCMS} repository.
	\end{description}
\end{enumerate}

\subsection{HCMS Access}

Access to a repository, content creation, content addition, and content modification are controlled by the repository owner(s).

\subsection{System Structure}

Employing the e-publication {\bf \small HFoF} as an example, we now describe content items and the way they are structured, implemented, and managed, and introduce the system functionality and its control within a {\bf \small HCMS} repository.

\subsubsection{The HCMS Repository}

Each {\bf \small HCMS} installation generates a stand alone repository containing the software required to support {\bf \small HCMS} functionality, and manage and update your repository content. Content items can be either generated in or added to this repository by the repository owner or an authorized content developer. Each content item may range in size from a single link to the addressable storage space known to a given {\bf \small HCMS}. External content can be hyperlinked to content within the {\bf \small HCMS}.

\subsubsection{Repository Location}

An {\bf \small HCMS} repository includes the directory structure of a given installation and the directory where your content items are maintained. Following installation of the {\bf \small HCMS}, the default location of your repository is:

\begin{verbatim}
    ~/neurospaces_project/<repository-name>/source/snapshots/0
\end{verbatim}

The source materials for individual content items are located in uniquely named folders/directories in this repository (at the end of the path in directory/folder ``0''). One of the directories/folders in the repository is named\,{\it NewDocument}. This default content item and the files it contains provide the templates for the creation of a new content item. The files of interest are called\,{\it descriptor.yml}, {\it NewDocument.tex}, and the sub-directory {\it figures}.

\subsubsection{Content Items}

A content item is a single document of any length. It may contain any number of photographs/illustrations you wish to include, and appropriate links to any other content hosted either by {\bf \small HCMS} or elsewhere on the internet.

{\bf Importantly, the addition of any content to a repository is an acknowledgement by the authorized content developer that they are the owner of the content they provide.} 

The smallest content item in an e-publication implementation of the {\bf \small HCMS} is a single replicated page of a physical book (e.g. {\bf \small HFoF}). Each such page is maintained as a separate content item in an {\bf \small HCMS} repository.

For more detailed information see \href{../developer-content/developer-content.tex}{\bf Developer\,Content}

\bibliographystyle{plain}
\bibliography{../tex/bib/g3-refs.bib}

\end{document}
