\documentclass[12pt]{article}
\usepackage{verbatim}
\usepackage[dvips]{epsfig}
\usepackage{color}
\usepackage{url}
\usepackage[colorlinks=true]{hyperref}
\usepackage{enumitem} % [noitemsep,nolistsep] compact itemize/enumerate etc
\usepackage[margin=2cm]{geometry}

% Typeface
\usepackage[condensed,math]{iwona}
\usepackage[T1]{fontenc}

\begin{document}

\section*{HETERARCH: Documentation}

{\bf Related Documentation:}
% start: userdocs-tag-replace-items related-do-nothing
% end: userdocs-tag-replace-items related-do-nothing

\section*{Heterarch}

\subsection*{Preamble}

The Heterarch Content Management System ({\bf \small HCMS}) is an open source software platform that supports distributed personalized digital content broadcasting. It is developed and maintained by {\bf Neurospaces\,Development} for {\bf Three\,Way\,Street}. 

%E-publication of the physical book {\bf{\emph{Head\,Full\,of\,Flames}}} ({\bf \small HFoF}) is used as a proof of concept for our novel approach to the structure and management of online content for the purpose of sophisticated online e-publication and community building.

The distinguishing feature of {\bf \small HCMS} is its use of modern techniques for the organization, structure, and recovery of digital content. Our approach is heterarchical and provides a powerful alternative to more traditional hierarchical methods of content management. It is the heterarchical nature of the {\bf \ small HCMS} that allows content recovery to be defined within ``explore'' and ``discover'' paradigms, as opposed to the still widely employed but limited alternative, the ``search'' paradigm.

A change in perspective from search to discover has many benefits for both developers and users. In particular, for the developer, it allows a far greater degree of automation for the many non-trivial but time consuming processes associated with sophisticated distributed personalized digital content broadcasting and content management. For the user, it provides a powerful information-based learning experience.

\subsection*{Document Purpose}

This document provides an introductory overview of the functionality of {\bf \small HCMS}.

\subsection*{Definitions}

There are two domains of human functionality associated with the {\bf \small HCMS}. Each of these domains is divided into two further sub-domains:

\begin{enumerate}[noitemsep,nolistsep] 
\item {\href{../hfof-developer/hfof-developer.tex}{\bf Developers:}}\\
The two sub-domains of developer functionality include:
	\begin{description}[noitemsep,nolistsep] 
	\item {\bf Content Developer}\\
	A content developer must obtain authorization from a repository owner before being able to directly access a given 
	repository.
		\begin{description}[noitemsep,nolistsep] 
		\item {\bf Authorized Content Developer:}\\
		An authorized content developer is any person given access to a repository for the purpose of creating, adding, 
		and/or modifying content. Two types of content developer are recognized:
			\begin{description}[noitemsep,nolistsep] 
			\item {\bf Repository owner(s):}\\
			By default, the owner(s) of a repository is/are authorized content developer(s).
			\item {\bf Authorized content developer(s):}\\
			The owner(s) of a repository can authorize any person the permissions required to access {\bf \small 
			HCMS} functionality of the given repository for the purpose of creating and/or adding content owned by the 
			authorized person.
			\end{description}
		\end{description}			
	\item {\bf Software Developer}\\
	A software developer extends {\bf \small HCMS} functionality by the creation and integration of new software and 
	functionality.
	\end{description}
\item {\bf Users:}\\
The two sub-classes of user include:
	\begin{description}[noitemsep,nolistsep] 
	\item {\bf Content User}\\
	A content user is any person who is not an Authorized Content Developer who accesses {\bf \small HCMS} content 
	through a web browser.
	\item {\bf Software User}\\
	A software user is any person who employs {\bf \small HCMS} software functionality to manage a given {\bf \small 
	HCMS} repository.
	\end{description}
\end{enumerate}

\subsection*{HCMS Access}

Access to a repository, content creation, content addition, and content modification are controlled by (a) repository owner(s).

\subsection*{System Structure}

We now describe content items and the way they are structured, implemented, and managed, and introduce the system functionality and its control within a {\bf \small HCMS} repository.

\subsubsection*{The HCMS Repository}

Each {\bf \small HCMS} installation generates a stand alone repository containing the software required to support {\bf \small HCMS} functionality, and manage and update repository content.\\
Content items can be either generated in or added to this repository by the repository owner or an authorized content developer.\\
Each content item may range in size from a single link to the addressable storage space known to a given {\bf \small HCMS}.\\
External content can be hyperlinked to content within a {\bf \small HCMS} repository.

\subsubsection*{Repository Location}

A {\bf \small HCMS} repository includes the directory structure of a given installation and the directory where content items are maintained. Following installation of a {\bf \small HCMS}, the default location of the repository is:

\begin{verbatim}
    ~/neurospaces_project/<repository-name>/source/snapshots/0
\end{verbatim}
The source materials for individual content items are located in uniquely named folders/directories in this repository (at the end of the path in directory/folder ``0''). One of the directories/folders in the repository is named\,{\it NewDocument}. This default content item and the files it contains provide the templates for the creation of a new content item. The files of interest are called\,{\it descriptor.yml}, {\it NewDocument.tex}, and the sub-directory {\it figures}.

\subsubsection*{Content Items}

An individual content item may have any size or any length. It may contain any number of photographs/illustrations and links to any other content located in either the same repository, another {\bf \small HCMS} repository, or elsewhere on the internet.\\
{\bf Importantly, the addition of any content to a workspace or repository is an acknowledgement by the authorized content developer of their ownership rights.}\\
For more detailed information see \href{../developer-content/developer-content.tex}{\bf Developer\,Content}.

\bibliographystyle{plain}
\bibliography{../tex/bib/g3-refs.bib}

\end{document}
